\chapter{CGI scripts}

\index{script!CGI}
{\color[rgb]{1 0 0}({\bf Warning:} CGI scripts without
appropriate safeguards can compromise your site's
security.  The scripts presented here are simple
examples and are not assured to be secure for actual
Web use.)}

\medskip

CGI scripts \cite{cgi} are scripts that reside on a
web server and can be run by a client (browser).  The
client accesses a CGI script by its URL, just as they
would a regular page.  The server, recognizing that the
URL requested is a CGI script, runs it.  How the server
recognizes certain URLs as scripts is up to the server
administrator.  For the purposes of this text, we will
assume that they are stored in a distinguished
directory called \path{cgi-bin}.  Thus, the script
\path{testcgi.scm} on the server \path{www.foo.org} would
be accessed as \path{http://www.foo.org/cgi-bin/testcgi.scm}.

The server runs the CGI script as the user \p{nobody},
who cannot be expected to have any
\p{PATH} knowledge (which is highly subjective
anyway).  Therefore the introductory magic line for a
CGI script written in Scheme needs to be a bit more
explicit than the one we used for ordinary Scheme
scripts.  Eg, the line

\q{
":";exec mzscheme -r $0 "$@"
}

\n implicitly assumes that there is a particular shell
(\p{bash}, say), and that there is a \p{PATH}, and that
\p{mzscheme} is in it.  For CGI scripts, we will need
to be more expansive:

\q{
#!/bin/sh
":";exec /usr/local/bin/mzscheme -r $0 "$@"
}

This gives fully qualified pathnames for the shell and
the Scheme executable.  The transfer of control from
shell to Scheme proceeds as for regular scripts.

\section{Example: Displaying environment variables}

Here is an example Scheme CGI script,
\path{testcgi.scm}, that outputs the settings of some
commonly used CGI environment variables.  This
information is returned as a new, freshly created, page
to the browser.  The returned page is simply whatever
the CGI script writes to its standard output.  This is
how CGI scripts talk back to whoever called them --- by
giving them a new page.

Note that the script first outputs the line

\p{
content-type: text/plain
}

\n {\em followed by a blank line}.  This is standard ritual
for a web server serving up a page.  These two lines
aren't part of what is actually displayed as the page.
They are there to inform the browser that the page being sent
is plain (ie, un-marked-up) text, so the browser can
display it appropriately.  If we were producing text
marked up in HTML, the
\p{content-type} would be
\p{text/html}.

The script \path{testcgi.scm}:

\scmfilename testcgi.scm

\scmdribble{
#!/bin/sh
":";exec /usr/local/bin/mzscheme -r $0 "$@"

;Identify content-type as plain text.

(display "content-type: text/plain") (newline)
(newline)

;Generate a page with the requested info.  This is
;done by simply writing to standard output.

(for-each
 (lambda (env-var)
   (display env-var)
   (display " = ")
   (display (or (getenv env-var) ""))
   (newline))
 '("AUTH_TYPE"
   "CONTENT_LENGTH"
   "CONTENT_TYPE"
   "DOCUMENT_ROOT"
   "GATEWAY_INTERFACE"
   "HTTP_ACCEPT"
   "HTTP_REFERER" ; [sic]
   "HTTP_USER_AGENT"
   "PATH_INFO"
   "PATH_TRANSLATED"
   "QUERY_STRING"
   "REMOTE_ADDR"
   "REMOTE_HOST"
   "REMOTE_IDENT"
   "REMOTE_USER"
   "REQUEST_METHOD"
   "SCRIPT_NAME"
   "SERVER_NAME"
   "SERVER_PORT"
   "SERVER_PROTOCOL"
   "SERVER_SOFTWARE"))
}

\p{testcgi.scm} can be called directly by opening it on
  a browser.  The URL is:

\p{
http://www.foo.org/cgi-bin/testcgi.scm
}

Alternately, \p{testcgi.scm} can occur as a link in an
HTML file, which you can click.  Eg,

\p{
... To view some common CGI environment variables, click
<a href="http://www.foo.org/cgi-bin/testcgi.scm">here</a>.
...
}

However \p{testcgi.scm} is launched, it will produce a
plain text page containing the settings of the
environment variables.  An example output:

\p{
AUTH_TYPE =
CONTENT_LENGTH =
CONTENT_TYPE =
DOCUMENT_ROOT = /home/httpd/html
GATEWAY_INTERFACE = CGI/1.1
HTTP_ACCEPT = image/gif, image/x-xbitmap, image/jpeg, image/pjpeg, */*
HTTP_REFERER =
HTTP_USER_AGENT = Mozilla/3.01Gold (X11; I; Linux 2.0.32 i586)
PATH_INFO =
PATH_TRANSLATED =
QUERY_STRING =
REMOTE_HOST = 127.0.0.1
REMOTE_ADDR = 127.0.0.1
REMOTE_IDENT =
REMOTE_USER =
REQUEST_METHOD = GET
SCRIPT_NAME = /cgi-bin/testcgi.scm
SERVER_NAME = localhost.localdomain
SERVER_PORT = 80
SERVER_PROTOCOL = HTTP/1.0
SERVER_SOFTWARE = Apache/1.2.4
}

\section{Example: Displaying selected environment variable}
\label{testcgi2}

\p{testcgi.scm} does not take any input from the user.
A more focused script would take an argument
environment variable from the user, and output the
setting of that variable and none else.  For this, we
need a mechanism for feeding arguments to CGI scripts.
The \p{form} tag of HTML provides this capability.
Here is a sample HTML page for this purpose:

\p{
<html>
<head>
<title>Form for checking environment variables</title>
</head>
<body>

<form method=get 
      action="http://www.foo.org/cgi-bin/testcgi2.scm">
Enter environment variable: <input type=text name=envvar size=30>
<p>

<input type=submit>
</form>

</body>
</html>
}

\n The user enters the desired environment variable (eg,
\p{GATEWAY_INTERFACE})  in the
textbox and clicks the submit button.  This causes all
the information in the form --- here, the setting of
the parameter \p{envvar} to the value
\p{GATEWAY_INTERFACE} --- to be collected and sent to
the CGI script identified by the \p{form}, viz,
\p{testcgi2.scm}.    The information can be sent in one
of two ways: (1) if the \p{form}'s \p{method=get} (the
default), the information is sent via the environment
variable called \p{QUERY_STRING}; (2) if the \p{form}'s
\p{method=post}, the information is available to the
CGI script at the latter's standard input port
(\p{stdin}).  Our form uses \p{QUERY_STRING}.

It is \p{testcgi2.scm}'s responsibility to extract the
information from
\p{QUERY_STRING}, and output the answer page
accordingly.

The information to the CGI script, whether arriving via
an environment variable or through \p{stdin}, is
formatted as a sequence of parameter/argument pairs.
The pairs are separated from each other by the \p{&}
character.  Within a pair, the parameter occurs first
and is separated from the argument by the \p{=}
character.  In this case, there is only
one parameter/argument pair, viz,
\p{envvar=GATEWAY_INTERFACE}.

The script \p{testcgi2.scm}:
\scmfilename testcgi2.scm

\scmdribble{
#!/bin/sh
":";exec /usr/local/bin/mzscheme -r $0 "$@"

(display "content-type: text/plain") (newline)
(newline)

;string-index returns the leftmost index in string s
;that has character c

(define string-index
  (lambda (s c)
    (let ((n (string-length s)))
      (let loop ((i 0))
        (cond ((>= i n) #f)
              ((char=? (string-ref s i) c) i)
              (else (loop (+ i 1))))))))

;split breaks string s into substrings separated by character c

(define split
  (lambda (c s)
    (let loop ((s s))
      (if (string=? s "") '()
          (let ((i (string-index s c)))
            (if i (cons (substring s 0 i)
                        (loop (substring s (+ i 1)
                                         (string-length s))))
                (list s)))))))

(define args
  (map (lambda (par-arg)
         (split #\= par-arg))
       (split #\& (getenv "QUERY_STRING"))))

(define envvar (cadr (assoc "envvar" args)))

(display envvar)
(display " = ")
(display (getenv envvar))

(newline)
}

\n Note the use of a helper procedure \q{split} to split
the \p{QUERY_STRING} into parameter/argument pairs
along the \p{&} character, and then splitting parameter
and argument along the \p{=} character.  (If we had
used the \p{post} method rather than \p{get}, we would
have needed to extract the parameters and arguments
from the standard input.)

The \p{<input type=text>} and \p{<input type=submit>}
are but two of the many different \p{input} tags
possible in an HTML \p{form}.  Consult \cite{cgi} for
the full repertoire.

\section{CGI script utilities}

In the example above, the parameter's name or the
argument it assumed did not themselves contain any
`\p{&}' or `\p{=}' characters.  In general, they may.
To accommodate such characters, and not have them be
mistaken for separators, the CGI argument-passing
mechanism treats all characters other than letters,
digits, and the underscore, as {\em special}, and
transmits them in an encoded form.  A space is encoded
as a `\p{+}'.  For other special characters, the
encoding is a three-character sequence, and consists of
`\p{%}' followed the special character's hexadecimal
code.  Thus, the character sequence `\p{20% + 30% =
50%, &c.}' will be encoded as

\p{
20%25+%2b+30%25+%3d+50%25%2c+%26c%2e
}

\n (Space become `\p{+}'; `\p{%}' becomes `\p{%25}';
`\p{+}' becomes `\p{%2b}'; `\p{=}' becomes `\p{%3d}';
`\p{,}' becomes `\p{%2c}'; `\p{&}' becomes `\p{%26}';
and `\p{.}' becomes `\p{%2e}'.)

Instead of dealing anew with the task of getting and
decoding the form data in each CGI script, it is
convenient to collect some helpful procedures into a
library file
\p{cgi.scm}.  \p{testcgi2.scm} can then be written
more compactly as

\scmfilename testcgi2a.scm

\scmdribble{
#!/bin/sh
":";exec /usr/local/bin/mzscheme -r $0 "$@"

;Load the cgi utilities

(load-relatve "cgi.scm")

(display "content-type: text/plain") (newline)
(newline)

;Read the data input via the form

(parse-form-data)

;Get the envvar parameter

(define envvar (form-data-get/1 "envvar"))

;Display the value of the envvar

(display envvar)
(display " = ")
(display (getenv envvar))
(newline)
}

\n This shorter CGI script uses two utility procedures
defined in \p{cgi.scm}. \q{parse-form-data} to read
the data supplied by the user via the form.  The data
consists of parameters and their associated values.
\q{form-data-get/1}  finds the  value associated with a
particular parameter.

\q{cgi.scm} defines a global table called
\q{*form-data-table*} to store form data.

\scmfilename cgi.scm

\scmdribble{
;Load our table definitions

(load-relative "table.scm")

;Define the *form-data-table*

(define *form-data-table* (make-table 'equ string=?))
}

\n An advantage of using a general mechanism such
as the \q{parse-form-data} procedure is that we can
hide the details of what \p{method} (\p{get} or
\p{post}) was used.

\scmdribble{
(define parse-form-data
  (lambda ()
    ((if (string-ci=? (or (getenv "REQUEST_METHOD") "GET") "GET")
         parse-form-data-using-query-string
         parse-form-data-using-stdin))))
}

\n The environment variable
\p{REQUEST_METHOD} tells which method was used to transmit
the form data.  If the method is \p{GET}, then the form
data was sent as the string available via another
environment variable, \p{QUERY_STRING}.  The auxiliary
procedure
\q{parse-form-data-using-query-string} is used to pick
apart \p{QUERY_STRING}:

\scmdribble{
(define parse-form-data-using-query-string
  (lambda ()
    (let ((query-string (or (getenv "QUERY_STRING") "")))
      (for-each
       (lambda (par=arg)
         (let ((par/arg (split #\= par=arg)))
           (let ((par (url-decode (car par/arg)))
                 (arg (url-decode (cadr par/arg))))
             (table-put! 
              *form-data-table* par
              (cons arg 
                    (table-get *form-data-table* par '()))))))
       (split #\& query-string)))))
}

\n The  helper procedure \q{split}, and {\em its} helper
\q{string-index}, are defined as in sec~\ref{testcgi2}.
As noted, the incoming form data is a sequence of
name-value pairs separated by \p{&}s.  Within each
pair, the name comes first, followed by an \p{=}
character, followed by the value.  Each name-value
combination is collected into a global table, the
\q{*form-data-table*}.

\scmwrite{
;string-index returns the leftmost index in string s
;that has character c

(define string-index
  (lambda (s c)
    (let ((n (string-length s)))
      (let loop ((i 0))
        (cond ((>= i n) #f)
              ((char=? (string-ref s i) c) i)
              (else (loop (+ i 1))))))))

;split breaks string s into substrings separated by character c

(define split
  (lambda (c s)
    (let loop ((s s))
      (if (string=? s "") '()
          (let ((i (string-index s c)))
            (if i (cons (substring s 0 i)
                        (loop (substring s (+ i 1)
                                         (string-length s))))
                (list s)))))))
}

\n Both name and value are encoded, so we
need to decode them using the \q{url-decode} procedure
to get their actual representation.

\scmdribble{
(define url-decode
  (lambda (s)
    (let ((s (string->list s)))
      (list->string
       (let loop ((s s))
         (if (null? s) '()
             (let ((a (car s)) (d (cdr s)))
               (case a
                 ((#\+) (cons #\space (loop d)))
                 ((#\%) (cons (hex->char (car d) (cadr d))
                              (loop (cddr d))))
                 (else (cons a (loop d)))))))))))
}

\n `\p{+}' is converted into space.  A triliteral of the
form `\p{%xy}'
is converted, using the procedure \q{hex->char} into
the character whose
ascii encoding is the hex number `\p{xy}'.

\scmdribble{
(define hex->char
  (lambda (x y)
    (integer->char
     (string->number (string x y) 16))))
}

\n We still need a form-data parser for the case where the
request method is \p{POST}.  The auxiliary procedure
\q{parse-form-data-using-stdin} does this.

\scmdribble{
(define parse-form-data-using-stdin
  (lambda ()
    (let* ((content-length (getenv "CONTENT_LENGTH"))
           (content-length
             (if content-length
                 (string->number content-length) 0))
           (i 0))
      (let par-loop ((par '()))
        (let ((c (read-char)))
          (set! i (+ i 1))
          (if (or (> i content-length)
                  (eof-object? c) (char=? c #\=))
              (let arg-loop ((arg '()))
                (let ((c (read-char)))
                  (set! i (+ i 1))
                  (if (or (> i content-length)
                          (eof-object? c) (char=? c #\&))
                      (let ((par (url-decode
                                   (list->string
                                     (reverse! par))))
                            (arg (url-decode
                                   (list->string
                                     (reverse! arg)))))
                        (table-put! *form-data-table* par
                          (cons arg (table-get *form-data-table*
                                      par '())))
                        (unless (or (> i content-length)
                                    (eof-object? c))
                          (par-loop '())))
                      (arg-loop (cons c arg)))))
              (par-loop (cons c par))))))))
}

\n The \p{POST} method sends form data via the script's
\p{stdin}.  The number of characters sent is placed in
the environment variable \p{CONTENT_LENGTH}.
\q{parse-form-data-using-stdin} reads the required
number of characters from \p{stdin}, and populates the
\q{*form-data-table*} as before, making sure to decode
the parameters' names and values.

It remains to retrieve the values for specific
parameters from the \q{*form-data-table*}.  Note that
the table associates a list with each parameter, in
order to accommodate the possibility of multiple values
for a parameter.  \q{form-data-get} retrieves all the
values assigned to a parameter.  If there is only one
value, it returns a singleton containing that value.

\scmdribble{
(define form-data-get
  (lambda (k)
    (table-get *form-data-table* k '())))
}

\n \q{form-data-get/1} returns the first (or most
  significant) value associated with a parameter.

\scmdribble{
(define form-data-get/1
  (lambda (k . default)
    (let ((vv (form-data-get k)))
      (cond ((pair? vv) (car vv))
            ((pair? default) (car default))
            (else "")))))
}

\n In our examples so far, the CGI script has generated
plain text.  Generally, though, we will want to
generate an HTML page.  It is not uncommon for a
combination of HTML form and CGI script to trigger a
series of HTML pages with forms.  It is also common to
code all the action corresponding to these various
forms in a single CGI script.  In any case, it is
helpful to have a utility procedure that writes out strings
in HTML format, ie, with the HTML special characters
encoded appropriately:

\scmdribble{
(define display-html
  (lambda (s . o)
    (let ((o (if (null? o) (current-output-port)
                 (car o))))
      (let ((n (string-length s)))
        (let loop ((i 0))
          (unless (>= i n)
            (let ((c (string-ref s i)))
              (display
               (case c
                 ((#\<) "&lt;")
                 ((#\>) "&gt;")
                 ((#\") "&quot;")
                 ((#\&) "&amp;")
                 (else c)) o)
              (loop (+ i 1)))))))))
}

\section{A calculator via CGI}

Here is an CGI calculator script, \p{cgicalc.scm}, that
exploits Scheme's arbitrary-precision arithmetic.

\scmfilename cgicalc.scm

\scmdribble{
#!/bin/sh
":";exec /usr/local/bin/mzscheme -r $0

;Load the CGI utilities
(load-relative "cgi.scm")

(define uhoh #f)

(define calc-eval
  (lambda (e)
    (if (pair? e)
        (apply (ensure-operator (car e))
               (map calc-eval (cdr e)))
        (ensure-number e))))

(define ensure-operator
  (lambda (e)
    (case e
      ((+) +)
      ((-) -)
      ((*) *)
      ((/) /)
      ((**) expt)
      (else (uhoh "unpermitted operator")))))

(define ensure-number
  (lambda (e)
    (if (number? e) e
        (uhoh "non-number"))))

(define print-form
  (lambda ()
    (display "<form action=\"")
    (display (getenv "SCRIPT_NAME"))
    (display "\">
  Enter arithmetic expression:<br>
  <input type=textarea name=arithexp><p>
  <input type=submit value=\"Evaluate\">
  <input type=reset value=\"Clear\">
</form>")))

(define print-page-begin
  (lambda ()
    (display "content-type: text/html

<html>
  <head>
    <title>A Scheme Calculator</title>
  </head>
  <body>")))

(define print-page-end
  (lambda ()
    (display "</body>
</html>")))

(parse-form-data)

(print-page-begin)

(let ((e (form-data-get "arithexp")))
  (unless (null? e)
    (let ((e1 (car e)))
      (display-html e1)
      (display "<p>
  =&gt;&nbsp;&nbsp;")
      (display-html
       (call/cc
        (lambda (k)
          (set! uhoh
                (lambda (s)
                  (k (string-append "Error: " s))))
          (number->string
           (calc-eval (read (open-input-string (car e))))))))
      (display "<p>"))))

(print-form)
(print-page-end)
}

%  A CGI script can accept arguments.  These may be
%specified as extra path information tagged on to the
%script's URL, or accumulated from various \p{<input>}
%statements inside an HTML \p{<form>}.

%input, or with extra path information on the URL, which
%is made available to the URL via an environment
%variable called \p{QUERY_STRING}.
